%%%%%%%%%%%%%%%%%%%%%%%%%%%%%%%%%%%%%%%%%
% Beamer Presentation
% LaTeX Template
% Version 1.0 (10/11/12)
%
% This template has been downloaded from:
% http://www.LaTeXTemplates.com
%
% License:
% CC BY-NC-SA 3.0 (http://creativecommons.org/licenses/by-nc-sa/3.0/)
%
%%%%%%%%%%%%%%%%%%%%%%%%%%%%%%%%%%%%%%%%%

%----------------------------------------------------------------------------------------
%   PACKAGES AND THEMES
%----------------------------------------------------------------------------------------

\documentclass{beamer}

\mode<presentation> {

%dencoding
%--------------------------------------
\usepackage[utf8]{inputenc}
\usepackage[T1]{fontenc}
%--------------------------------------
\usepackage{graphicx}
%Portuguese-specific commands
%--------------------------------------
\usepackage[portuguese]{babel}
%--------------------------------------

%Hyphenation rules
%--------------------------------------
\usepackage{hyphenat}
\hyphenation{mate-mática recu-perar}
%--------------------------------------

%--------------------------------------
\usepackage{textpos}
%--------------------------------------
\fboxsep0pt
%--------------------------------------

% --verticalmente centralizado----
\usepackage{array,booktabs}% http://ctan.org/pkg/{array,booktabs}
% ---------------


\bibliographystyle{plain}

% The Beamer class comes with a number of default slide themes
% which change the colors and layouts of slides. Below this is a list
% of all the themes, uncomment each in turn to see what they look like.

%\usetheme{default}
%\usetheme{AnnArbor}
%\usetheme{Antibes}
%\usetheme{Bergen}
%\usetheme{Berkeley}
%\usetheme{Berlin}
%\usetheme{Boadilla}
%\usetheme{CambridgeUS}
%\usetheme{Copenhagen}
%\usetheme{Darmstadt}
%\usetheme{Dresden}
%\usetheme{Frankfurt}
%\usetheme{Goettingen}
%\usetheme{Hannover}
%\usetheme{Ilmenau}
%\usetheme{JuanLesPins}
%\usetheme{Luebeck}
%\usetheme{Madrid}
%\usetheme{Malmoe}
%\usetheme{Marburg}
\usetheme{Montpellier}
%\usetheme{PaloAlto}
%\usetheme{Pittsburgh}
%\usetheme{Rochester}
%\usetheme{Singapore}
%\usetheme{Szeged}
%\usetheme{Warsaw}



% As well as themes, the Beamer class has a number of color themes
% for any slide theme. Uncomment each of these in turn to see how it
% changes the colors of your current slide theme.

%\usecolortheme{albatross}
%\usecolortheme{beaver}
%\usecolortheme{beetle}
%\usecolortheme{crane}
\usecolortheme{dolphin}
%\usecolortheme{dove}
%\usecolortheme{fly}
%\usecolortheme{lily}
%\usecolortheme{orchid}
%\usecolortheme{rose}
%\usecolortheme{seagull}
%\usecolortheme{seahorse}
%\usecolortheme{whale}
%\usecolortheme{wolverine}

%\setbeamertemplate{footline} % To remove the footer line in all slides uncomment this line
\setbeamertemplate{footline} % To replace the footer line in all slides with a simple slide count uncomment this line

\setbeamertemplate{footline}{
\leavevmode%
  \hbox{%
  \begin{beamercolorbox}[wd=.4\paperwidth,ht=2.25ex,dp=1ex,center]{author in head/foot}%
    \usebeamerfont{author in head/foot}\insertshortauthor
  \end{beamercolorbox}%
  \begin{beamercolorbox}[wd=.6\paperwidth,ht=2.25ex,dp=1ex,center]{title in head/foot}%
    \usebeamerfont{title in head/foot}\insertshorttitle\hspace*{6em}
    \insertframenumber{} / \inserttotalframenumber\hspace*{1ex}
  \end{beamercolorbox}}%
  \vskip0pt
}

%\setbeamertemplate{navigation symbols}{} % To remove the navigation symbols from the bottom of all slides uncomment this line
}

\usepackage{graphicx} % Allows including images
\usepackage{booktabs} % Allows the use of \toprule, \midrule and \bottomrule in tables

% letras riscadas em cima
\usepackage[normalem]{ulem}
\usepackage[colorlinks]{hyperref}
\usefonttheme{serif}
% -----------------------------------------------------



%%----------------------------------------------------------------------------------------
%   TITLE PAGE
%----------------------------------------------------------------------------------------

%cores
%--------------------------------------
\newcommand{\azul}[1]{\textcolor{blue}{#1}}

\newcommand{\vermelho}[1]{\textcolor{red}{#1}}

\newcommand{\verde}[1]{\textcolor{green}{#1}}

\newcommand{\preto}[1]{\textcolor{black}{#1}}

\setbeamertemplate{background}{\tikz[overlay,remember picture]\node[opacity=0.4]at (current page.center){\includegraphics[width=2cm]{me(1)}};}


\usepackage{tikz}
\usepackage{kantlipsum}
%--------------------------------------
\title[Apresentação do Portfólio]{Portfólio} % The short title appears at the bottom of every slide, the full title is only on the title page

\author[Rodrigo Hermont Ozon]{
Rodrigo Hermont Ozon\\
} % Your name

\institute[UFPR] % Your institution as it will appear on the bottom of every slide, may be shorthand to save space
{
Histórico de Entregas e Conquistas\\ % Your institution for the title page


\medskip

\textcolor{blue}{\url{https://www.linkedin.com/in/rodrigohermontozon}}

\vspace{.25cm}
\textcolor{blue}{\url{http://lattes.cnpq.br/3532649625879285}
}}


\date{\today} % Date, can be changed to a custom date

% numerais romanos
\makeatletter
\newcommand*{\rom}[1]{\expandafter\@slowromancap\romannumeral #1@}
\makeatother
\begin{document}

\begin{frame}
\titlepage % Print the title page as the first slide
\end{frame}

\newcommand{\sumario}{
\begin{frame}[allowframebreaks]{Outline}
\frametitle{Sumário} % Table of contents slide, comment this block out to remove it
\tableofcontents % Throughout your presentation, if you choose to use \section{} and \subsection{} commands, these will automatically be printed on this slide as an overview of your presentation
\end{frame}
}

\sumario

%----------------------------------------------------------------------------------------
%   PRESENTATION SLIDES
%-----------------------------------------------------------\
\section{Apresentação Pessoal}
\begin{frame}
\frametitle{Apresentação pessoal}

\centering
\includegraphics[widht=2.5cm,height=2.5cm]{imagens/me.jpg}

\footnotesize
\begin{flushleft}

   \textit{Rodrigo Hermont Ozon, 36 anos, economista, ocupei os seguintes cargos}: 
   
   \begin{itemize}
 
     \item<1-4> \textit{Data Scientist}

     \item<2-4> Coordenador de Projetos, Captação de Recursos
   
     \item<3-4> Pesquisador e Professor

   \item<4-4> Consultoria Econômica
   \end{itemize}

   
\end{flushleft}
    
\end{frame}

%------------------------------------------------
\subsection{Resumo dos cargos ocupados}

\begin{frame}
\frametitle{Resumo dos serviços prestados}

\footnotesize
Já trabalhei nas seguintes áreas:

\small
\begin{itemize}
    \item Mercado Financeiro
    \item Pesquisa científica
    \item Terceiro Setor 
    \item Indústria 
    \item Tecnologia (\textit{start up})
    \item Professor universitário
\end{itemize}


\end{frame}
%------------------------------------------------

%------------------------------------------------
\begin{frame}
\frametitle{Resumo dos serviços prestados}

\footnotesize
Como \textit{Data Scientist}:


\begin{itemize}
    \item<1-4> Construção de indicadores de perfomance (KPIs), indicadores de risco (KRIs), relatórios gerenciais dinâmicos em tempo real.
    \item<2-4> Construção de \textit{dashboard} de visualização em tempo real (\textit{Business Intelligence}) de dados financeiros (ativos financeiros) e informações contábeis para a alta gestão 
    \item<3-4> Visualização de dados estatísticos da economia paranaense (visão de cadeias produtivas industriais)
    \item<4-4> Construção de Índice de Desenvolvimento Empresarial com base em dados secundários da economia;
\end{itemize}

\end{frame}

%------------------------------------------------
\begin{frame}
\frametitle{Resumo dos serviços prestados}

\footnotesize
Como coordenador de projetos:
\begin{itemize}
\item Trabalhei com captação de recursos para execução do projeto submetendo proposta a diversos órgãos de financiamento conforme suas exigências (geralmente editais/chamadas específicas) 
\item Submissão, planejamento e reprogramação de plano de projetos (implantação de ferramental de gestão de projetos)
\item Fui avaliador dos projetos que as empresas candidatavam ao edital Senai/Sesi de Inovação. Também fui avaliador no PAPPE/Subvenção parceria IBQP/Sebrae
\item Professor de pós-graduação na FESP no MBA em Gerenciamento de Projetos na disciplina de Gerenciamento de Integração de Projetos metodologia PmBok/PMI
\end{itemize}

\end{frame}
%------------------------------------------------
\begin{frame}
\frametitle{Resumo dos serviços prestados}

\footnotesize
Como pesquisador, professor e consultor:

\begin{itemize}
    \item<1-3> Fui pesquisador bolsista da Fundação Araucária e FIEP sobre cadeias produtivas industriais paranaenses, isso me deu \textit{know how} para trabalhar com estudos/pesquisa de mercado de alto nível quando fiz mestrado em desenvolvimento econômico
    
    \item<2-3> Fui professor das disciplinas na graduação: Mercados Financeiros, Economia Monetária e Financeira, Métodos Quantitativos, Controle Estatístico de Processos, Administração Financeira, Indicadores de Desempenho da Indústria.  
    
    \item<3-3> Como consultor viajei os estados brasileiros contribuindo para a implantação e melhoria do projeto ID-MPE (mais detalhes a seguir \textcolor{blue}{\ref{idmpe}}) para os Sebraes UF. Também faço consultoria econômica para empresas e pessoas físicas com planos de investimentos;
\end{itemize}

\end{frame}
%------------------------------------------------------------
\section{Entregas: Projetos e trabalhos feitos por onde atuei}
\begin{frame}
\frametitle{Projetos e trabalhos feitos}

\footnotesize
Consulte meus projetos que estão disponíveis na \textit{web}:

Como \textit{Data Scientist}:

\begin{itemize}
    \item<1-4> \textcolor{blue}{\href{http://www.leg.ufpr.br/doku.php/projetos:ehlers:volprev}{Modelagem de Volatilidade em Séries Financeiras (Laboratório de Estatística e Geoinformação/UFPR)}} 
    
    \item<2-4> \textcolor{blue}{\href{https://rhoportolio.blogspot.com/2019/09/salarios-de-admissao-por-cargo.html}{Estudo de Mercado para parametrização salarial para contratação de profissionais (visualização feita em \textit{Power BI} com os microdados do CAGED/Min. Trabalho e Emprego)}} 
    
    \item<3-4> \textcolor{blue}{\href{http://sites.pr.sebrae.com.br/leigeral/wp-content/uploads/sites/35/2014/02/Cartilha_IDMPE_PR_-_junho_2011.pdf}{Índice de Desenvolvimento Municipal das Micro e Pequenas Empresas, 2011 (SEBRAE/IBQP)}}\label{idmpe}
    \item<4-4> \textcolor{blue}{\href{http://sites.pr.sebrae.com.br/leigeral/wp-content/uploads/sites/35/2014/09/IDME_2013_web.pdf}{IDMPE (edição 2009/2010 para Sebrae PR)\footnote{\tiny Pode se considerar aqui como \\ Data Scientist/Consultor e Coordenador de Projetos} }}
\end{itemize}

\pause *As demais publicações dos projetos descritos no meu LinkedIn não foram publicados na web por decisão de seus custeadores.

\end{frame}
%------------------------------------------------------------
\begin{frame}
\frametitle{Projetos e trabalhos feitos}

\footnotesize
Dos projetos que coordenei e executei e não estão disponíveis na \textit{web} destaco:
\begin{itemize}
    \item<1-3> No Grupo Bitcoin Banco (GBB): Análise e Indicadores de Risco (área de atendimento para GBB), Automação dos relatórios de Desempenho Gerencial da área de Atendimento do GBB; Indicador de Desempenho e Performance da Área de Atendimento GBB; Relatório \textit{realtime} dos dados do \textit{e-commerce} Get4Bit oriundos do \textit{Google Analitycs}; Visualização de Dados do Setor de Compras do GBB (\href{https://www.linkedin.com/in/rodrigohermontozon/}{\textcolor{blue}{vide perfil no meu Linkedin)}}\footnote{\tiny Dados confidenciais e utilizados somente para divulgação interna.}
    \item<2-3> Na FIEP: Fiz análises setoriais de cadeias produtivas industriais paranaenses, contribui para a \textcolor{blue}{\href{http://www.fiepr.org.br/para-empresas/estudos-economicos/sondagem-industrial-1-20654-238938.shtml}{Pesquisa de Sondagem Industrial}} e apoiei as negociações coletivas salariais de alguns sindicatos filiados. Também fiz captação de recursos submetendo projetos a instituições de financiamento como Banco do Brasil, FINEP, PNUD/ONu etc.
    \item<3-3> No \textcolor{blue}{\href{http://ibqp.org.br/}{IBQP}} além do IDMPE fui voluntário/avaliador do \textcolor{blue}{\href{http://www.ibqp.org.br/projetos/movimento-parana-competitivo/sobre-mpc/}{Movimento Paraná Competitivo}} da \textcolor{blue}{\href{http://www.fnq.org.br/}{Fundação Nacional da Qualidade}} 
\end{itemize}



\end{frame}
%-------------------------------------------------
\section{Perfis Públicos (posts) de meus trabalhos}
\begin{frame}
\frametitle{Perfis Públicos}

\footnotesize
Acesse minhas publicações (visualizações de dados e linhas de códigos) nos seguintes endereços:


\begin{itemize}
    \item No GitHub: \textcolor{blue}{\href{https://github.com/rhozon}{https://github.com/rhozon (aqui está disponível este portfolio na linguagem \LaTeX \hspace{.05cm} utilizando o pacote Beamer)}}
    \item Na galeria do Tableau Public: \textcolor{blue}{\href{https://public.tableau.com/profile/rodrigo.h.ozon#!/}{https://public.tableau.com/profile/rodrigo.h.ozon#!/}}
    \item Protótipo de visualização de dados para a FIEP: \textcolor{blue}{\href{http://fiepdesenvolvimento.blogspot.com/}{http://fiepdesenvolvimento.blogspot.com/}}
    \item Perfil no DrivenData \textcolor{blue}{\href{https://www.drivendata.org/users/rhozon/}{https://www.drivendata.org/users/rhozon/}}
\end{itemize}


\end{frame}
%------------------------------------------------------------
\section{Habilidades e know how}
\begin{frame}
\frametitle{Formação Acadêmica}

\footnotesize
\begin{itemize}
    \item \textcolor{blue}{\href{http://uniaodavitoria.unespar.edu.br/ensino/graduacao/matematica}{Licenc. e Hab. Plena em Matemática pela Universidade Estadual do Paraná (2003)}} 
    \item \textcolor{blue}{\href{http://www.economia.ufpr.br/}{Bacharel em Ciências Econômicas pela Universidade Federal do Paraná (2008)}}
    \item \textcolor{blue}{\href{http://www.sociaisaplicadas.ufpr.br/portal/depecon/pos-graduacao/}{Mestre em Desenvolvimento Econômico pela UFPR (2011)}}
    \item \textcolor{blue}{\href{https://www.fatemi.com.br/}{Teologia Ministerial (FATEMI/UNICESUMAR) (2020)}}
    \item \textcolor{blue}{\href{https://universidade.up.edu.br/pos-graduacao/programa-de-especializacao-docente-no-ensino-da-ciencias/}{Especialização em Estratégias de Ensino e Aprendizagem na Educação Superior (Universidade Positivo trancado em 2010)}}
\end{itemize}


\end{frame}
%----------------------------------------------------------------


%----------------------------------------------------------

\begin{frame}
\frametitle{Certificações}

\footnotesize
\begin{itemize}
    \item Certificação profissional em Ciência de Dados pela IBM (em conclusão até dez/2019) 
    
    
    \item Capelania Assistencial pela Cruz Vermelha Internacional
\end{itemize}

\end{frame}
%---------------------------------------------------------
\begin{frame}
\frametitle{Principais cursos}

\footnotesize
\begin{itemize}
    \item Econometria Básica Aplicada pela USP (2019)
    \item \textit{Bitcoin and CryptoCurrency Technologies} pela Universidade de Princeton (2019)
    \item \textcolor{blue}{\textit{Experimentando com o \href{https://www.ufpr.br/portalufpr/eventos/curso-de-extensao-economia-e-complexidade-modelagem-baseada-em-agentes-com-lsd-sera-promovido-de-17-a-19-de-julho/}{Laboratory for Simulation Dynamics}, fronteiras e perspectivas} (UFPR)} (2009)
    \item Indicadores de Gestão pelo IBQP (2010)
    \item Modelos Matemáticos em finanças I e II pela UFPR (2008)
    \item Instrumentos Financeiros Aplicados à Decisões de Mercado (2005)
    \item Softwares Livres de Gestão pelo IBQP (2008)
\end{itemize}
*Para consultar todos os cursos feitos, \textcolor{blue}{\href{http://lattes.cnpq.br/3532649625879285}{consulte meu CV na Plataforma Lattes do CNPq}}

\end{frame}
%----------------------------------------------------------------

\begin{frame}{Produção Científica}

\footnotesize
\textcolor{blue}{\href{http://lattes.cnpq.br/3532649625879285}{Segundo os indicadores de produtividade do CNPq, conforme a Plataforma Lattes, detenho os seguintes resultados:}}
\begin{itemize}
    \item \textbf{1} Artigo completo publicado em periódico
    \item \textbf{1} Livro publicado/organizado ou edições
    \item \textbf{13} Textos em jornais de notícias/revistas
    \item \textbf{1} Artigo aceito para publicação
    \item \textbf{2} Apresentações de trabalhos
    \item \textbf{1} Outras publicações bibliográficas
    \item \textbf{1} Assessoria/consultoria parecerista
    \item \textbf{3} Trabalhos técnicos
    \item \textbf{15} Demais tipos de produção técnica
    \item \textbf{1} Participação em bancas de comissões julgadoras 
    \item \textbf{1} Participação em eventos, congressos, exposições e feiras
    \item \textbf{5} Orientações e supervisões em andamento
    \item \textbf{1} Capitulo de livro escrito \textcolor{blue}{\href{http://ibqp.org.br/wp-content/uploads/2016/10/Empreendedorismo-no-Brasil-2011-Relat\%C3\%83\%C2\%B3rio.pdf}{(Pesquisa \textit{Global Enterpreneuship Monitor (GEM))}}}
\end{itemize}
    
\end{frame}
%------------------------------------------------------------
%----------------------------------------------------------------

\begin{frame}{Conhecimentos Computacionais}
\footnotesize
Linguagens:
\begin{itemize}
    \item \textcolor{blue}{\href{http://www.leg.ufpr.br/~paulojus/embrapa/Rembrapa/Rembrapase35.html}{LaTeX+R (Sweave)}}, VBA (Excel), SQL, LaTeX2HTML, Matlab, Stata, SPSS, Eviews
\end{itemize}

Bibliotecas e frameworks:
\begin{itemize}
    \item Matplotlib, Gnuplot, GeoGebra e Data.table
\end{itemize}

Ferramentas e Plataformas
\begin{itemize}
    \item Microsoft Power BI, Tableau Public, Power Point, AWS, Data Studio, Git e Github, RStudio, Visual Studio, Excel
\end{itemize}
Armazenamento de dados, infraestrutura e cloud
\begin{itemize}
    \item SQL Server, MySQL, Oracel, JSON, Azure
\end{itemize}
Atividades e Conceitos
\begin{itemize}
\item ETL, criação de webcrawler, processamento em tempo real, modelagem de banco de dados, visualização de dados, arquitetura de dados, estatísticas, formalismo e modelagem matemática, teste AB
\end{itemize}

\end{frame}
%-------------------------------------------------------


%------------------------------------------------.

\section{Contato}
\begin{frame}
\frametitle{Contato}

\begin{columns}[t]
\begin{column}{5cm}


\textit{\textcolor{blue}{\href{https://api.whatsapp.com/send?phone=5541988382904&text=Ol\%C3\%A1\%20Rodrigo\%2C\%20vi\%20seu\%20portf\%C3\%B3lio\%20no\%20linkedin\%2C\%20tenho\%20interesse\%20em\%20conhecer\%20um\%20pouco\%20mais\%20do\%20seu\%20trabalho.\%20Podemos\%20conversar\%20a\%20respeito\%20\%3F}{Me adicione no whatsapp:}}}


\vspace{.5cm}

\textit{\textcolor{blue}{\href{https://api.whatsapp.com/send?phone=5541988382904&text=Ol\%C3\%A1\%20Rodrigo\%2C\%20vi\%20seu\%20portf\%C3\%B3lio\%20no\%20linkedin\%2C\%20tenho\%20interesse\%20em\%20conhecer\%20um\%20pouco\%20mais\%20do\%20seu\%20trabalho.\%20Podemos\%20conversar\%20a\%20respeito\%20\%3F}{+55 41 98838-2904}}} 


\end{column}
\begin{column}{5cm}

\textit{\textcolor{blue}{\href{mailto:rodrigoozon@yahoo.com.br}{Me envie e-mail:}}}

\vspace{.5cm}
\textit{\textcolor{blue}{\href{mailto:rodrigoozon@yahoo.com.br}{rodrigoozon@yahoo.com.br}}} \\

\end{column}
\end{columns}

\end{frame}

%----------------------------------------




%------------------------------------------------------------
\end{document}

